% Stephen Weinreich
% Murmann Mixed-Signal Group, Stanford University
% Jan 2020

\documentclass[conference]{IEEEtran}
% Stephen Weinreich
% Murmann Mixed-Signal Group, Stanford University
% Jan 2020

% Citations
\usepackage{cite}
% Math
\usepackage{amsmath}
\interdisplaylinepenalty=2500 % enable line breaks within multiline equations
% Alignment
\usepackage{array}
% IEEEtran contains the IEEEeqnarray family of commands that can be used to
% generate multiline equations as well as matrices, tables, etc., of high
% quality.
% Subfigure
\ifCLASSOPTIONcompsoc
 \usepackage[caption=false,font=normalsize,labelfont=sf,textfont=sf]{subfig}
\else
 \usepackage[caption=false,font=footnotesize]{subfig}
\fi

% correct bad hyphenation here
\hyphenation{op-tical net-works semi-conduc-tor}



% *** MY ADDITIONS ***
% Improved tables
\usepackage{tabularx}
\usepackage{multirow}
\usepackage[flushleft]{threeparttable}

% Unicode characters (in citations)
\DeclareUnicodeCharacter{2052}{\%}
\DeclareUnicodeCharacter{00B5}{$\mu$}

% 1st, 2nd, ...
\usepackage[super]{nth}

% Dummy text
\usepackage{lipsum}

% Figure generation
\usepackage{pgfplots}
\pgfplotsset{compat=1.9}
\usetikzlibrary{pgfplots.groupplots}
\usepackage{alphalph}
\usepackage[RPvoltages]{circuitikz}
\tikzstyle{every path}=[line cap=round,line join=round]
\usetikzlibrary{calc}

% Custom command for table Y/N
\usepackage{pifont}
\newcommand{\n}[1][]{\phantom{$^{#1}$}-$^{#1}$}
\newcommand{\y}[1][]{\phantom{$^{#1}$}\ding{51}$^{#1}$}

% Reduce spacing around figures
\addtolength{\textfloatsep}{-0.3em}

% Reduce spacing around equations
\setlength{\abovedisplayskip}{5pt}
\setlength{\belowdisplayskip}{5pt}

\begin{document}
% Set IEEEtran bibliography rules before the first citation
\bstctlcite{IEEEexample:BSTcontrol}


\title{Cadwiki Overleaf Example}


% Make sure you maintain proper spacing if your initial submission needs a redacted author list.
% This is set up for one line of authors with three affiliations
% \author{\IEEEauthorblockN{(redacted author list)}
% \IEEEauthorblockA{(redacted affiliation list)}
% \IEEEauthorblockA{}
% \IEEEauthorblockA{}
% }
\author{\IEEEauthorblockN{Stephen Weinreich}
\IEEEauthorblockA{Department of Electrical Engineering, Stanford University, California, USA}
}

\maketitle


\begin{abstract}
\lipsum[1]
\end{abstract}


\section{Introduction}
\label{section_intro}

\lipsum[2-5]

\begin{figure}[!t]
\centering
\begin{tikzpicture}
\begin{axis}[
	% Plot dimensions
	width=3.4in,
	height=2in,
	% Set font sizes to match your figure caption
	tick label style={font=\footnotesize},
	label style={font=\footnotesize},
	legend style={font=\footnotesize},
	% Axis labels, limits, tick marks, and grid lines. Limits and ticks will be auto-configurd if commented out.
	xlabel={Time (s)},
	ylabel={Amplitude (V)},
	xmin=0, xmax=2,
	ymin=-2, ymax=2,
	% xtick={-8, -4, 0, 4, 8},
	% ytick={-5, 0, 5},
	ymajorgrids=true,
	xmajorgrids=true,
	grid style=dashed,
	% Configure the legend
	% This sets the legend at the top center, just outside the plot. "Anchor" means that the southern point (i.e. bottom center) of the legend bounding box is set to the given coordinates.
	legend style={at={(0.50,1.00)}, anchor=south}, 
	% If -1, it will use a single row and as many columns as needed.
	legend columns=2,
	% Text alignment
	legend cell align={left},
	% Hide the border around the legend
	legend style={draw=none},
]
\addplot[color=black,mark=none,style=very thick] table[col sep=comma,header=true,x=x,y=sq] {data/data_prep_example.csv};
\addplot[color=blue,mark=none,style=very thick] table[col sep=comma,header=true,x=x,y=sq1] {data/data_prep_example.csv};
\addplot[color=red,mark=none,style=very thick] table[col sep=comma,header=true,x=x,y=sq3] {data/data_prep_example.csv};
\addplot[color=black,mark=none,style=densely dashed] table[col sep=comma,header=true,x=x,y=sq5] {data/data_prep_example.csv};
\legend{Square wave\cite{Muratore2019}\phantom{--},Fundamental\phantom{--},Three harmonics\phantom{--},Five harmonics}
\end{axis}
\end{tikzpicture}
\caption{A plot.}
\label{fig_single}
\vspace{-0.4em}
\end{figure}

\begin{figure}[!t]
\centering
\begin{tikzpicture}
\makeatletter
\begin{groupplot}[
	% Plot dimensions.
    width=3.4in,
    height=3in,
    group style={
		% Name doesn't seem to matter (can probably be referenced somewhere)
        group name="analysis",
        group size=2 by 2,
        % Use shared x and y axes
        x descriptions at=edge bottom,
        y descriptions at=edge left,
		% Set spacing between plots
        horizontal sep=0.1in,
        vertical sep=0.1in,
    },
	% Adds (a), (b), etc labels to each figure.
    every axis/.append style={
        extra description/.code={
            \node at (0.06,0.06) {\footnotesize{(\alphalph{\pgfplots@group@current@plot})}};
        }
    },
	% Set font sizes to match your figure caption
    footnotesize,
    tick label style={font=\footnotesize},
    label style={font=\footnotesize},
    legend style={font=\footnotesize},
	% Axis labels, limits, tick marks, and grid lines. Limits and ticks will be auto-configurd if commented out.
	xlabel={Time (s)},
	ylabel={Amplitude (V)},
	xmin=0, xmax=2,
	ymin=-2, ymax=2,
	% xtick={-8, -4, 0, 4, 8},
	% ytick={-5, 0, 5},
	ymajorgrids=true,
	xmajorgrids=true,
	grid style=dashed,
	% Draw markers along with lines, so subsequent lines are on top of previous markers
    clip mode=individual,
	% Configure the legend
	% This sets the legend at the top center, just outside the plot. "Anchor" means that the southern point (i.e. bottom center) of the legend bounding box is set to the given coordinates. In this case we're going to create the legend on the top-left plot, hence the given coordinates.
    legend style={at={(1.04,1.00)}, anchor=south},
	% If -1, it will use a single row and as many columns as needed.
    legend columns=2,
	% Text alignment
    legend cell align={left},
	% Hide the border around the legend
    legend style={draw=none},
    legend style={inner sep=0.1em},
	% Reverse legend order, relative to plotting stack order
    reverse legend,
]
\makeatother
\nextgroupplot
\addplot[color=black,mark=none,style=very thick] table[col sep=comma,header=true,x=x,y=sq] {data/data_prep_example.csv};
\addplot[color=blue,mark=none,style=very thick] table[col sep=comma,header=true,x=x,y=sq1] {data/data_prep_example.csv};
\addplot[color=red,mark=none,style=very thick] table[col sep=comma,header=true,x=x,y=sq3] {data/data_prep_example.csv};
\addplot[color=black,mark=none,style=densely dashed] table[col sep=comma,header=true,x=x,y=sq5] {data/data_prep_example.csv};
\legend{Square wave\phantom{--},Fundamental\phantom{--},Three harmonics\phantom{--},Five harmonics}
\nextgroupplot
\addplot[color=blue,mark=none,style=very thick] table[col sep=comma,header=true,x=x,y=sq1] {data/data_prep_example.csv};
\nextgroupplot
\addplot[color=red,mark=none,style=very thick] table[col sep=comma,header=true,x=x,y=sq3] {data/data_prep_example.csv};
\nextgroupplot
\addplot[color=black,mark=none,style=densely dashed] table[col sep=comma,header=true,x=x,y=sq5] {data/data_prep_example.csv};
\end{groupplot}
\end{tikzpicture}
\swDecreaseFigSpacing
\caption{Multiple plots.}
\label{fig_multi}
\vspace{-0.4em}
\end{figure}


\section{Lorem ipsum}
\label{section_lorem_ipsum}

\lipsum[6-8]

\begin{figure}[!t]
\centering
\tikzset{C/.append style={/tikz/circuitikz/bipoles/length=0.4cm}}
\tikzset{R/.append style={/tikz/circuitikz/bipoles/length=0.4cm}}
\tikzset{every rxantenna node/.append style={/tikz/circuitikz/bipoles/length=0.7cm}}
\begin{circuitikz}[
    scale=0.25,
    every node/.style={scale=0.8},
    every circ node/.style={scale=0.7},
    ]
    \tikzstyle{every node}=[font=\scriptsize]
    \tikzstyle{every path}=[line width=0.6pt]
	
    %%% Set various positions
    \def\xin{-8}; % mixer input
    \def\yin{0}; % mixer input
    \def\xcl{-1.5}; % C_L
    \def\xrl{0}; % R_L
    \def\yA{4.5}; % vbb0/3
    \def\yB{1.5}; % vbb1/2

    %%% Antenna
    \draw (\xin,\yin) node[circ]{} -- ++(-1,0) node[rxantenna,xscale=-1]{};

    %% Switches
    \draw (\xin,\yin) -- (\xin, \yA)              -- ++(1,0) to[switch,l=\raisebox{0.2em}{SW0}] ++(4,0) -- (0,\yA);
    \draw         (\xin, \yB) node[circ]{} -- ++(1,0) to[switch,l=\raisebox{0.2em}{SW1}] ++(4,0) -- (0,\yB);
    \draw         (\xin,-\yB) node[circ]{} -- ++(1,0) to[switch,l=\raisebox{0.2em}{SW2}] ++(4,0) -- (0,-\yB);
    \draw (\xin,\yin) -- (\xin,-\yA)              -- ++(1,0) to[switch,l=\raisebox{0.2em}{SW3}] ++(4,0) -- (0,-\yA);
    %% Baseband
    \draw (\xcl, +\yA) node[circ]{} -- ++(0,-0.5) to[C] ++(0,-1.0) node[sground,scale=0.3,rotate=0]{};
    \draw (\xcl, +\yB) node[circ]{} -- ++(0,-0.5) to[C] ++(0,-1.0) node[sground,scale=0.3,rotate=0]{};
    \draw (\xcl, -\yB) node[circ]{} -- ++(0,-0.5) to[C] ++(0,-1.0) node[sground,scale=0.3,rotate=0]{};
    \draw (\xcl, -\yA) node[circ]{} -- ++(0,-0.5) to[C] ++(0,-1.0) node[sground,scale=0.3,rotate=0]{};
    \draw (\xrl, +\yA)              -- ++(0,-0.5) to[R] ++(0,-1.0) node[sground,scale=0.3,rotate=0]{};
    \draw (\xrl, +\yB)              -- ++(0,-0.5) to[R] ++(0,-1.0) node[sground,scale=0.3,rotate=0]{};
    \draw (\xrl, -\yB)              -- ++(0,-0.5) to[R] ++(0,-1.0) node[sground,scale=0.3,rotate=0]{};
    \draw (\xrl, -\yA)              -- ++(0,-0.5) to[R] ++(0,-1.0) node[sground,scale=0.3,rotate=0]{};

    %%% Timing diagram
    \tikzstyle{every path}=[line width=1.2pt]
    \draw (5,-3) coordinate (lo);
    \draw (lo)++(0,4.50) ++(0,0.5) node[]{\footnotesize SW0} ++(1.75,-0.5) -- ++(0,0) -- ++(0,1) -- ++(1,0) -- ++(0,-1) -- ++(3,0);
    \draw (lo)++(0,3.00) ++(0,0.5) node[]{\footnotesize SW1} ++(1.75,-0.5) -- ++(1,0) -- ++(0,1) -- ++(1,0) -- ++(0,-1) -- ++(2,0);
    \draw (lo)++(0,1.50) ++(0,0.5) node[]{\footnotesize SW2} ++(1.75,-0.5) -- ++(2,0) -- ++(0,1) -- ++(1,0) -- ++(0,-1) -- ++(1,0);
    \draw (lo)++(0,0.00) ++(0,0.5) node[]{\footnotesize SW3} ++(1.75,-0.5) -- ++(3,0) -- ++(0,1) -- ++(1,0) -- ++(0,-1) -- ++(0,0);
\end{circuitikz}
\swDecreaseFigSpacing
\caption{A CircuiTikZ example.}
\label{fig_mixer}
\vspace{-0.4em}
\end{figure}

\begin{figure}[!t]
\centering
\tikzset{C/.append style={/tikz/circuitikz/bipoles/length=0.5cm}}
\begin{circuitikz}[
    scale=0.25,
    every node/.style={scale=0.8},
    every circ node/.style={scale=0.7},
    every ocirc node/.style={scale=0.7},
    ]
    \tikzstyle{every node}=[font=\scriptsize]
    \tikzstyle{every path}=[line width=0.6pt,line cap=round,line join=round]
    
    %%% Set various positions
    % Input, mixer, output, opamp
    \def\xin{-16}; % mixer input
    \def\xcl{-8}; % C_L
    \def\xout{7}; % output
    \def\xop{0}; % opamp
    \def\yop{5.8}; % opamp
    % Feedback
    \def\xfbbox{-3.5}; % fb box ll corner (centered at 0)
    \def\yfbbox{-1.5};
    \def\fboffseta{1}; % connection x-offsets from fb box
    \def\fboffsetb{2};
    \def\fbconna{1.2}; % y-connections to fb box
    \def\fbconnb{0.4};
    

    
    %%% Q-path
    % opamp
    \draw (\xop,-\yop) node[fd op amp] (opamp) {};
    \node at (opamp) {\normalsize A\phantom{AAA}}; % hack to get spacing right
    \newdimen\yopp; \pgfextracty{\yopp}{\pgfpointanchor{opamp}{+}};
    \newdimen\yopn; \pgfextracty{\yopn}{\pgfpointanchor{opamp}{-}};
    % switches
    \draw (\xin,0) -- (\xin,\yopp)          -- ++(1,0) to[switch,l=\raisebox{0.2em}{SW3}] ++(4,0) -- (opamp.+);
    \draw         (\xin,\yopn) node[circ]{} -- ++(1,0) to[switch,l=\raisebox{0.2em}{SW1}] ++(4,0) -- (opamp.-);
    % capacitors
    \draw (\xcl, \yopn) node[circ]{} coordinate(x) to[C,l_=\sub{C}{L}\hspace*{-0.5em}] (x |- opamp)
        node[circ]{} -- ++(0.5,0) node[sground,scale=0.3,rotate=90]{};
    \draw (\xcl, \yopp) node[circ]{} coordinate(x) to[C,l=\sub{C}{L}\hspace*{-0.5em}] (x |- opamp);
    % connections to feedback
    \draw (\xfbbox, -\fbconna) -- ++(-\fboffseta, 0) coordinate(x) -- (x |- opamp.+) node[circ]{};
    \draw (\xfbbox, -\fbconnb) -- ++(-\fboffsetb, 0) coordinate(x) -- (x |- opamp.-) node[circ]{};
    \draw (-\xfbbox, -\fbconnb) -- ++(\fboffseta, 0) coordinate(x) -- (x |- opamp.out +) node[circ]{};
    \draw (-\xfbbox, -\fbconna) -- ++(\fboffsetb, 0) coordinate(x) -- (x |- opamp.out -) node[circ]{};
    % output
    \draw (opamp.out -) -- (\xout, \yopp)  node[ocirc]{};
    \draw (opamp.out +) -- (\xout, \yopn)  node[ocirc]{};
    \node[align=center] at (\xout, -\yop) {+\\\sub{V}{Q}\\-};
    
    %%% I-path 
    % opamp
    \draw (\xop,\yop) node[fd op amp] (opamp) {};
    \node at (opamp) {\normalsize A\phantom{AAA}}; % hack to get spacing right
    \newdimen\yopp; \pgfextracty{\yopp}{\pgfpointanchor{opamp}{+}};
    \newdimen\yopn; \pgfextracty{\yopn}{\pgfpointanchor{opamp}{-}};
    % switches
    \draw (\xin,0) -- (\xin,\yopn)          -- ++(1,0) to[switch,l=\raisebox{0.2em}{SW0}] ++(4,0) -- (opamp.-);
    \draw         (\xin,\yopp) node[circ]{} -- ++(1,0) to[switch,l=\raisebox{0.2em}{SW2}] ++(4,0) -- (opamp.+);
    % capacitors
    \draw (\xcl, \yopn) node[circ]{} coordinate(x) to[C,l_=\sub{C}{L}\hspace*{-0.5em}] (x |- opamp)
        node[circ]{} -- ++(0.5,0) node[sground,scale=0.3,rotate=90]{};
    \draw (\xcl, \yopp) node[circ]{} coordinate(x) to[C,l=\sub{C}{L}\hspace*{-0.5em}] (x |- opamp);
    % connections to feedback
    \draw (\xfbbox, \fbconna) -- ++(-\fboffseta, 0) coordinate(x) -- (x |- opamp.+) node[circ]{};
    \draw (\xfbbox, \fbconnb) -- ++(-\fboffsetb, 0) coordinate(x) -- (x |- opamp.-) node[circ]{};
    \draw (-\xfbbox, \fbconnb) -- ++(\fboffseta, 0) coordinate(x) -- (x |- opamp.out +) node[circ]{};
    \draw (-\xfbbox, \fbconna) -- ++(\fboffsetb, 0) coordinate(x) -- (x |- opamp.out -) node[circ]{};
    % output
    \draw (opamp.out -) -- (\xout, \yopp)  node[ocirc]{};
    \draw (opamp.out +) -- (\xout, \yopn)  node[ocirc]{};
    \node[align=center] at (\xout, \yop) {+\\\sub{V}{I}\\-};
    
    %%% Feedback box
    \draw [thick] (-\xfbbox,-\yfbbox) rectangle (\xfbbox,\yfbbox) node[pos=.5,align=center] {complex\\feedback\cite{Andrews2010a}};

	%%% Labels
	\draw (\xin, 0)++(-1,0.5)       node[currarrow]{} +(0,0) -- ++(-1.2,0) -- ++(0,1.2) to[short, l_=\hspace*{-0.3em}\sub{Z}{in}] ++(0,0.6);
	\draw (\xin, \yopn)++(0.5,0.5)    node[currarrow]{} +(0,0) -- ++(-1.2,0) -- ++(0,1.2) to[short, l_=\hspace*{-0.3em}\sub{Z}{path}] ++(0,0.6);
	\draw (\xcl, \yopn)++(-1.5,0.5) node[currarrow]{} +(0,0) -- ++(-1.2,0) -- ++(0,1.2) to[short, l_=\hspace*{-0.3em}\sub{Z}{BB}] ++(0,0.6);
	\draw (\xcl, \yopn)++(1.75,0.5) node[currarrow]{} +(0,0) -- ++(-1.2,0) -- ++(0,1.2) to[short, l_=\hspace*{-0.3em}\sub{Z}{L}] ++(0,0.6);

    %% Antenna
    \draw (\xin,0) node[circ]{} -- ++(-3,0) to[/tikz/circuitikz/bipoles/length=1cm,european resistor, l_=\footnotesize\raisebox{0.3em}{\sub{Z}{RF}}] ++(-4,0) -- ++(-1,0) coordinate (left)
        -- ++(0,-1) to[/tikz/circuitikz/bipoles/length=0.8cm,sV,l=\hspace*{-0.3em}\sub{V}{RF}] ++(0,-5) node[/tikz/circuitikz/bipoles/length=0.8cm,sground]{};
	
	%%% Antenna impedance inset
	% box and text
    \draw [dashed] (left)++(-2.0,2.8) coordinate (boxLL) rectangle ++(7.6,7.3) coordinate (boxUR);
    \draw [dashed] (left)++(1,1) -- (boxLL);
    \draw [dashed] (left)++(4.7,1) -- (boxUR |- boxLL);
    \draw ($(boxLL)!0.5!(boxUR)$ |- boxLL)++(0,-3.2) coordinate (za_txt);
    \node at (za_txt) {\footnotesize (Section \ref{sec_analysis})};
    % RLC
    \draw (boxLL)++(0.5,3.0) node[ocirc]{} coordinate(za1) to[/tikz/circuitikz/bipoles/length=0.5cm,R,l=\raisebox{0.2em}{50}] ++(3,0) coordinate(N1) -- ++(0,0.9)
        to[/tikz/circuitikz/bipoles/length=0.6cm,american inductor,l=\raisebox{0.2em}{0.5n}] ++(3,0) -- ++(0,-0.9) coordinate(N2) -- ++(0.6,0) coordinate(za2) node[ocirc]{};
    \draw (N1) node[circ]{} -- ++(0,-0.9) to[/tikz/circuitikz/bipoles/length=0.3cm,C,l=\raisebox{0.2em}{0.5p}] ++(3,0) -- (N2) node[circ]{};
    % R
    \draw (za1 |- boxLL)++(0,5.9) coordinate(za3) node[ocirc]{}  to[/tikz/circuitikz/bipoles/length=0.5cm,R,l=\raisebox{0.2em}{50}] (za2 |- za3) node[ocirc]{};
\end{circuitikz}
\caption{Another CircuiTikZ example.}
\label{fig_rx}
\vspace{-0.4em}
\end{figure}


\subsection{Lorem ipsum}
\lipsum[18-20]

\begin{table}[!t]
\centering
\begin{threeparttable}
% increase table row spacing, adjust to taste
\renewcommand{\arraystretch}{1.25}
\setlength\tabcolsep{0.24em}
\caption{Example of a Table With Checkboxes}
\label{table_example}
\centering
\begin{tabular}{l|ccccc}
\vspace{-0.3em}
Method & Item 1 & Item 2 & Item 3 & Item 4 & Item 5 \tabularnewline
\hline
First method \cite{Hameed2018,Hameed2015} & \y[1] & \y & \y[2] & \n & \n \tabularnewline
Second method \cite{Muratore2019} & \n & \y & \y[2] & \y & \n \tabularnewline
Third method \cite{Kwon2018,Maas2003} & \y[3] & \n & \n & \y & \n \tabularnewline
Fourth method \cite{Andrews2012,Andrews2010a,Hammler2014} & \y[1] & \y & \y[2] & \y & \y \tabularnewline
Fifth method \cite{Klumperink2017} & \y & \y & \n & \y & \y \tabularnewline
\end{tabular}
\begin{tablenotes}
\scriptsize
  \item $^1$A tabular footnote.
  \item $^2$Another tabular footnote.
  \item $^3$Yet another tabular footnote.
\end{tablenotes}
\end{threeparttable}
\end{table}


\section{Conclusion}
\label{section_concl}
\lipsum[35]


\section*{Acknowledgment}
\lipsum[50][1-2]


\IEEEtriggeratref{5}
\bibliographystyle{IEEEtran}
\bibliography{IEEEabrv,bib/IEEEtransBSTCTL,bib/cadwiki_example}

\end{document}