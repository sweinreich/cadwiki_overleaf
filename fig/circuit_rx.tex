\tikzset{C/.append style={/tikz/circuitikz/bipoles/length=0.5cm}}
\begin{circuitikz}[
    scale=0.25,
    every node/.style={scale=0.8},
    every circ node/.style={scale=0.7},
    every ocirc node/.style={scale=0.7},
    ]
    \tikzstyle{every node}=[font=\scriptsize]
    \tikzstyle{every path}=[line width=0.6pt,line cap=round,line join=round]
    
    %%% Set various positions
    % Input, mixer, output, opamp
    \def\xin{-16}; % mixer input
    \def\xcl{-8}; % C_L
    \def\xout{7}; % output
    \def\xop{0}; % opamp
    \def\yop{5.8}; % opamp
    % Feedback
    \def\xfbbox{-3.5}; % fb box ll corner (centered at 0)
    \def\yfbbox{-1.5};
    \def\fboffseta{1}; % connection x-offsets from fb box
    \def\fboffsetb{2};
    \def\fbconna{1.2}; % y-connections to fb box
    \def\fbconnb{0.4};
    

    
    %%% Q-path
    % opamp
    \draw (\xop,-\yop) node[fd op amp] (opamp) {};
    \node at (opamp) {\normalsize A\phantom{AAA}}; % hack to get spacing right
    \newdimen\yopp; \pgfextracty{\yopp}{\pgfpointanchor{opamp}{+}};
    \newdimen\yopn; \pgfextracty{\yopn}{\pgfpointanchor{opamp}{-}};
    % switches
    \draw (\xin,0) -- (\xin,\yopp)          -- ++(1,0) to[switch,l=\raisebox{0.2em}{SW3}] ++(4,0) -- (opamp.+);
    \draw         (\xin,\yopn) node[circ]{} -- ++(1,0) to[switch,l=\raisebox{0.2em}{SW1}] ++(4,0) -- (opamp.-);
    % capacitors
    \draw (\xcl, \yopn) node[circ]{} coordinate(x) to[C,l_=\sub{C}{L}\hspace*{-0.5em}] (x |- opamp)
        node[circ]{} -- ++(0.5,0) node[sground,scale=0.3,rotate=90]{};
    \draw (\xcl, \yopp) node[circ]{} coordinate(x) to[C,l=\sub{C}{L}\hspace*{-0.5em}] (x |- opamp);
    % connections to feedback
    \draw (\xfbbox, -\fbconna) -- ++(-\fboffseta, 0) coordinate(x) -- (x |- opamp.+) node[circ]{};
    \draw (\xfbbox, -\fbconnb) -- ++(-\fboffsetb, 0) coordinate(x) -- (x |- opamp.-) node[circ]{};
    \draw (-\xfbbox, -\fbconnb) -- ++(\fboffseta, 0) coordinate(x) -- (x |- opamp.out +) node[circ]{};
    \draw (-\xfbbox, -\fbconna) -- ++(\fboffsetb, 0) coordinate(x) -- (x |- opamp.out -) node[circ]{};
    % output
    \draw (opamp.out -) -- (\xout, \yopp)  node[ocirc]{};
    \draw (opamp.out +) -- (\xout, \yopn)  node[ocirc]{};
    \node[align=center] at (\xout, -\yop) {+\\\sub{V}{Q}\\-};
    
    %%% I-path 
    % opamp
    \draw (\xop,\yop) node[fd op amp] (opamp) {};
    \node at (opamp) {\normalsize A\phantom{AAA}}; % hack to get spacing right
    \newdimen\yopp; \pgfextracty{\yopp}{\pgfpointanchor{opamp}{+}};
    \newdimen\yopn; \pgfextracty{\yopn}{\pgfpointanchor{opamp}{-}};
    % switches
    \draw (\xin,0) -- (\xin,\yopn)          -- ++(1,0) to[switch,l=\raisebox{0.2em}{SW0}] ++(4,0) -- (opamp.-);
    \draw         (\xin,\yopp) node[circ]{} -- ++(1,0) to[switch,l=\raisebox{0.2em}{SW2}] ++(4,0) -- (opamp.+);
    % capacitors
    \draw (\xcl, \yopn) node[circ]{} coordinate(x) to[C,l_=\sub{C}{L}\hspace*{-0.5em}] (x |- opamp)
        node[circ]{} -- ++(0.5,0) node[sground,scale=0.3,rotate=90]{};
    \draw (\xcl, \yopp) node[circ]{} coordinate(x) to[C,l=\sub{C}{L}\hspace*{-0.5em}] (x |- opamp);
    % connections to feedback
    \draw (\xfbbox, \fbconna) -- ++(-\fboffseta, 0) coordinate(x) -- (x |- opamp.+) node[circ]{};
    \draw (\xfbbox, \fbconnb) -- ++(-\fboffsetb, 0) coordinate(x) -- (x |- opamp.-) node[circ]{};
    \draw (-\xfbbox, \fbconnb) -- ++(\fboffseta, 0) coordinate(x) -- (x |- opamp.out +) node[circ]{};
    \draw (-\xfbbox, \fbconna) -- ++(\fboffsetb, 0) coordinate(x) -- (x |- opamp.out -) node[circ]{};
    % output
    \draw (opamp.out -) -- (\xout, \yopp)  node[ocirc]{};
    \draw (opamp.out +) -- (\xout, \yopn)  node[ocirc]{};
    \node[align=center] at (\xout, \yop) {+\\\sub{V}{I}\\-};
    
    %%% Feedback box
    \draw [thick] (-\xfbbox,-\yfbbox) rectangle (\xfbbox,\yfbbox) node[pos=.5,align=center] {complex\\feedback\cite{Andrews2010a}};

	%%% Labels
	\draw (\xin, 0)++(-1,0.5)       node[currarrow]{} +(0,0) -- ++(-1.2,0) -- ++(0,1.2) to[short, l_=\hspace*{-0.3em}\sub{Z}{in}] ++(0,0.6);
	\draw (\xin, \yopn)++(0.5,0.5)    node[currarrow]{} +(0,0) -- ++(-1.2,0) -- ++(0,1.2) to[short, l_=\hspace*{-0.3em}\sub{Z}{path}] ++(0,0.6);
	\draw (\xcl, \yopn)++(-1.5,0.5) node[currarrow]{} +(0,0) -- ++(-1.2,0) -- ++(0,1.2) to[short, l_=\hspace*{-0.3em}\sub{Z}{BB}] ++(0,0.6);
	\draw (\xcl, \yopn)++(1.75,0.5) node[currarrow]{} +(0,0) -- ++(-1.2,0) -- ++(0,1.2) to[short, l_=\hspace*{-0.3em}\sub{Z}{L}] ++(0,0.6);

    %% Antenna
    \draw (\xin,0) node[circ]{} -- ++(-3,0) to[/tikz/circuitikz/bipoles/length=1cm,european resistor, l_=\footnotesize\raisebox{0.3em}{\sub{Z}{RF}}] ++(-4,0) -- ++(-1,0) coordinate (left)
        -- ++(0,-1) to[/tikz/circuitikz/bipoles/length=0.8cm,sV,l=\hspace*{-0.3em}\sub{V}{RF}] ++(0,-5) node[/tikz/circuitikz/bipoles/length=0.8cm,sground]{};
	
	%%% Antenna impedance inset
	% box and text
    \draw [dashed] (left)++(-2.0,2.8) coordinate (boxLL) rectangle ++(7.6,7.3) coordinate (boxUR);
    \draw [dashed] (left)++(1,1) -- (boxLL);
    \draw [dashed] (left)++(4.7,1) -- (boxUR |- boxLL);
    \draw ($(boxLL)!0.5!(boxUR)$ |- boxLL)++(0,-3.2) coordinate (za_txt);
    \node at (za_txt) {\footnotesize (Section \ref{section_lorem_ipsum})};
    % RLC
    \draw (boxLL)++(0.5,3.0) node[ocirc]{} coordinate(za1) to[/tikz/circuitikz/bipoles/length=0.5cm,R,l=\raisebox{0.2em}{50}] ++(3,0) coordinate(N1) -- ++(0,0.9)
        to[/tikz/circuitikz/bipoles/length=0.6cm,american inductor,l=\raisebox{0.2em}{0.5n}] ++(3,0) -- ++(0,-0.9) coordinate(N2) -- ++(0.6,0) coordinate(za2) node[ocirc]{};
    \draw (N1) node[circ]{} -- ++(0,-0.9) to[/tikz/circuitikz/bipoles/length=0.3cm,C,l=\raisebox{0.2em}{0.5p}] ++(3,0) -- (N2) node[circ]{};
    % R
    \draw (za1 |- boxLL)++(0,5.9) coordinate(za3) node[ocirc]{}  to[/tikz/circuitikz/bipoles/length=0.5cm,R,l=\raisebox{0.2em}{50}] (za2 |- za3) node[ocirc]{};
\end{circuitikz}
\swDecreaseFigSpacing